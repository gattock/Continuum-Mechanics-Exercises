\documentclass[a4paper,11pt]{article}
\usepackage{amsmath}
\usepackage{physics}
\usepackage {graphicx}
\usepackage{geometry}
\usepackage{listings}
\usepackage{verbatim}
\usepackage{alltt}
\geometry{left=2.5cm,right=2.5cm,top=2.5cm,bottom=2.5cm}
	\begin{document}
\title{Continuum mechanics Homework 2}
\author{Enrico Di Lavore}
\date{27 November 2021}
\maketitle

Are given the equation of motion by the linear mapping $\chi (\mathbf(X))$
\[ \mathbf{x}=\chi(\mathbf{X})=(a X_1+b X_2+ c X_3) \mathbf{e_1}+
(d X_1+e X_2+ f X_3) \mathbf{e_2}+
(gX_1+h X_2+ l X_3) \mathbf{e_3}  \]
and its coefficients a=-1; b=1; c=0; d=3; e=3; f=0; g=0; h=0; l=1; (Version n.3)\\
It is also given the stress tensor $\mathbf{T}$ [MPa]
\[\mathbf{T}= \begin{pmatrix}
   140 &  160 &  300\\
   160  & 300 &  460\\
   300  & 460  & 600 \end{pmatrix} \]

	\section{Ex 1: Deformation gradient \textbf{F} tensor}
We explicit the deformation gradient tensor  $\mathbf{F}$ 

\[ \mathbf{F}=F(i,j)=\pdv{x_i}{X_j}= \begin{pmatrix}
\alpha    &  \beta          &     \kappa \\
 \beta    &  \alpha+\beta    &  \beta+\kappa \\
 \kappa  &   \beta+\kappa  & \alpha+\beta+\kappa \end{pmatrix}= \begin{pmatrix}
    -1 &    1  &   0\\
     3  &   3   &  0\\
     0  &   0    & 1 \end{pmatrix}\]
     
     \section{Ex 2: $2^{nd}$ Piola-Kirchhoff stress tensor $\mathbf{S}$}
We can compute the $2^{nd}$ Piola-Kirchhoff stress tensor $\mathbf{S}$ by
\[ \mathbf{S}= \mathbf{F^{-1}} \cdot \mathbf{T^0}=J \, \mathbf{F^{-1}} \cdot \mathbf{T} \cdot \mathbf{(F^{-1})^T}= \begin{pmatrix}
        -100   &      160    &     440\\
         160    &    -420    &   -1360\\
         440    &  -1360    &   -3600\end{pmatrix}\]
Where: \\
$\mathbf{F^{-1}}$ is the inverse of the tensor $\mathbf{F}$
\[ \mathbf{F^{-1}}= \begin{pmatrix}
         -0.5    &  0.16667       &     0\\
          0.5     & 0.16667        &    0\\
            0      &      0          &  1  \end{pmatrix}\]

$\mathbf{T^0}$ is the $1^{st}$ Piola-Kirchhoff stress tensor
\[ \mathbf{T^0}=J \,\mathbf{T} \, \mathbf{(F^{-1})^T}= \begin{pmatrix}
          260   &     -580   &    -1800\\
         180     &   -780    &   -2760\\
         440     &  -1360    &   -3600  \end{pmatrix}\]
         
\section{Ex 3: Compute von Mises stress for the tensor $\mathbf{T}$}
As first we compute the trace of the $2^{nd}$ order tensor $\mathbf{T}$ as the sum of its eigen-values:
\[  tr(\mathbf{T})=\sum_{i=1}^{3} \lambda_i=1057+41-58=1040 \]
Then we compute the hydrostatic $\mathbf{T}_{hyd}$ and deviatoric $\mathbf{T}_{dev}$ components of $\mathbf{T}$ as:
\[  T_{hyd}(i,j)=  \frac {tr(\mathbf{T}) \delta_{ij}}{3}= \begin{pmatrix}
       346.67      &      0     &       0\\
            0     &  346.67     &       0\\
            0       &     0   &    346.67 \end{pmatrix}  \]
And we compute the deviatoric component by:
\[ \mathbf{T}_{dev}= \mathbf{T}-\mathbf{T}_{hyd} = \begin{pmatrix}
      -206.67    &      160      &    300\\
          160    &  -46.667    &      460\\
          300     &     460  &     253.33 \end{pmatrix}  \]
          
Then it follows the computation of von Mises stress : 
\[ \sigma_{vM}=\sqrt{1.5 \, \mathbf{T}_{dev} : \mathbf{T}_{dev} }=\sqrt{1.5 \, \sum_{i,j=1}^{3} [T_{dev}(i,j)]^2 } \Rightarrow\] 
$ \; \Rightarrow \; \sigma_{vM}=1070.1$ [MPa]

\section{Ex 4: Compute strain tensor $\mathbf{\epsilon}$ }
We are given the value of Young modulus $E=70$[GPa] and the Poisson coefficient $\nu=0.3$. \\
We can compute the strain tensor $\mathbf{\epsilon}$ by
\[  \epsilon(i,j)=\frac{1}{E}[(1+\nu) \, \sigma_{ij}-\nu \, \delta_{ij} \, \sigma_{kk}] \; \Rightarrow \;  \mathbf{\epsilon}=\frac{1}{E}[(1+\nu ) \mathbf{T}-\nu \, \mathbf{I} \, tr(\mathbf{T})]\]
\[ \mathbf{\epsilon}= \begin{pmatrix}
      -1.8571    &   2.9714  &     5.5714\\
       2.9714    &   1.1143   &    8.5429\\
       5.5714    &   8.5429   &    6.6857 \end{pmatrix}\]
And we can decompose $\mathbf{\epsilon}$ in its hydrostatic and deviatoric components:

\[  \epsilon_{hyd}(i,j)=  \frac{tr(\mathbf{\epsilon}) \delta_{ij}}{3}= \begin{pmatrix}
         1.981      &      0      &      0\\
            0     &   1.981       &     0\\
            0       &     0    &    1.981 \end{pmatrix}  \]
\[ \mathbf{\epsilon}_{dev}=\mathbf{\epsilon}-\mathbf{\epsilon}_{hyd} = \begin{pmatrix}
       -3.8381   &    2.9714   &    5.5714\\
       2.9714   &  -0.86667   &    8.5429\\
       5.5714   &    8.5429   &    4.7048  \end{pmatrix}  \]

\subsection{Show that it holds the relation $T_{dev}(i,j)=2 \, G \, \epsilon_{dev}(i,j)$ }
We compute the shear modulus in function of Young modulus and Poisson coefficient\\ $G=\frac{E}{2\,(1+\nu)}=26.92$[GPa]\\
We verify the given relation by computing $T_{dev}-2 \, G \, \epsilon_{dev}$ and we see that we get a zero matrix.\\
Indeed the following terms are equal:
\[ \mathbf{T}_{dev}= \begin{pmatrix}
      -206.67    &      160      &    300\\
          160    &  -46.667    &      460\\
          300     &     460  &     253.33 \end{pmatrix}  \]
\[ 2 \, G \, \epsilon_{dev} = \begin{pmatrix}
      -206.67    &      160  &        300\\
          160      &-46.667    &      460\\
          300       &   460      &  253.33 \end{pmatrix}  \]
 
\section{Stiffness tensor computation and component explicitation}
We compute the tensor $\mathbf{C}$ by the expression:
\[ C(i,j,k,l)=\frac{E}{1+\nu}[0.5(\delta_{ik} \, \delta_{jl}+\delta_{jk} \, \delta_{il})+
\frac{\nu}{1-2\, \nu}\delta_{ij} \, \delta_{kl}]  \]
In particular, the ''slice'' of $4^{th}$ order tensor \textbf{C} that we are looking for is 
\[ C(:,:,2,3) = \begin{pmatrix}
            0     &       0     &       0\\
            0      &      0      & 26.923\\
            0      & 26.923     &       0 \end{pmatrix} \]
            
In particular $ C(2,3,2,3)=26.923$
\newpage

 
	 \section{Appendix: Cmd Window and Script}
 
 \subsection{Results from Matlab command window:}
 \begin{verbatim}

T =
   140   160   300
   160   300   460
   300   460   600
Ex1: compute deformation gradient tensor
F =
    -1     1     0
     3     3     0
     0     0     1
--------------------------------------------------------------------------
Ex2:
invF =
         -0.5      0.16667            0
          0.5      0.16667            0
            0            0            1
J =
    -6
------------------------
First PK stress tensor
T0 =
         260        -580       -1800
         180        -780       -2760
         440       -1360       -3600
Second PK stress tensor
S =
        -100         160         440
         160        -420       -1360
         440       -1360       -3600
--------------------------------------------------------------------------
Ex3:Von Mises stress computation
eigvec =
       0.4743      0.80989      0.34514
      0.60606     -0.58473      0.53924
     -0.63854     0.046586      0.76818
eigval =
      -59.437            0            0
            0       41.739            0
            0            0       1057.7
trT =
        1040
T_hyd =
       346.67            0            0
            0       346.67            0
            0            0       346.67
T_dev =
      -206.67          160          300
          160      -46.667          460
          300          460       253.33
sigma_vm =
       1070.1
--------------------------------------------------------------------------
Ex4: Strain tensor Epsilon computation by Hooke's law
E =
    70
nu =
          0.3
Eps =
      -1.8571       2.9714       5.5714
       2.9714       1.1143       8.5429
       5.5714       8.5429       6.6857
Show the relation between deviators Tˆ_ij = 2 G Epsˆ_ij
T_dev =
      -206.67          160          300
          160      -46.667          460
          300          460       253.33
Eps_hyd =
        1.981            0            0
            0        1.981            0
            0            0        1.981
Eps_dev =
      -3.8381       2.9714       5.5714
       2.9714     -0.86667       8.5429
       5.5714       8.5429       4.7048
G =
       26.923
2 G Eps_dev=
ans =
      -206.67          160          300
          160      -46.667          460
          300          460       253.33
TestVar =
     0
The relation between deviators is verified
--------------------------------------------------------------------------
Ex5: Stiffness tensor computation
C(:,:,1,1) =
       94.231            0            0
            0       40.385            0
            0            0       40.385
C(:,:,2,1) =
            0       26.923            0
       26.923            0            0
            0            0            0
C(:,:,3,1) =
            0            0       26.923
            0            0            0
       26.923            0            0
C(:,:,1,2) =
            0       26.923            0
       26.923            0            0
            0            0            0
C(:,:,2,2) =
       40.385            0            0
            0       94.231            0
            0            0       40.385
C(:,:,3,2) =
            0            0            0
            0            0       26.923
            0       26.923            0
C(:,:,1,3) =
            0            0       26.923
            0            0            0
       26.923            0            0
C(:,:,2,3) =
            0            0            0
            0            0       26.923
            0       26.923            0
C(:,:,3,3) =
       40.385            0            0
            0       40.385            0
            0            0       94.231
--------------------------------------------------------------------------
C(2,3,2,3) is equal to
ans =
       26.923
--------------------------------------------------------------------------
Hello World

\end{verbatim}

\newpage

\subsection{Matlab script:}
\begin{verbatim}
clc; clear all; %ver3
format shortG %cut decimals as needed
format compact %cmd window compact out
%axis equal
a=-1; b=1; c=0; d=3; e=3; f=0; g=0; h=0; l=1; 
aa=140; bb=160; kk=300; %alpha,beta,kappa
dash="--------------------------------------------------------------------------";
T=[aa bb kk ; bb aa+bb bb+kk ; kk bb+kk aa+bb+kk]
disp("Ex1: compute deformation gradient tensor")
F=[ a b c; d e f ; g h l]
disp(dash)
disp("Ex2:")%ex2:compute 2nd piola-kirchhoff stress tensor S
invF=inv(F)
J=det(F)
disp("------------------------")
disp("First PK stress tensor")
T0=J * T * invF' %first piola-kirch stress tens
disp("Second PK stress tensor")
S=invF *T0 %second p-k 

disp(dash) %ex3 Von Mises
disp('Ex3:Von Mises stress computation')
[eigvec,eigval]=eig(T)
trT=trace(T)
T_hyd=zeros(3,3);
for i=1:3
T_hyd(i,i)=trT/3;
end
T_hyd
T_dev=T-T_hyd

temp=0;
for i=1:3 %computation of sigma prime
    for j=1:3
        temp=temp+T_dev(i,j)^2;
    end
end
sigma_vm=(3/2*temp)^0.5

disp(dash) %ex4 
disp("Ex4: Strain tensor Epsilon computation by Hooke's law")
E=70 %[GPa]
nu=0.3
Eps=[(1+nu)*T-nu*eye(3)*trT]/E
disp("Show the relation between deviators Tˆ_ij = 2 G Epsˆ_ij")
T_dev
for i=1:3
Eps_hyd(i,i)=trace(Eps)/3;
end
Eps_hyd
Eps_dev=Eps-Eps_hyd
G=E/2/(1+nu)
disp("2 G Eps_dev=")
2*G*Eps_dev
TestVar=det(T_dev-2*G*Eps_dev)
if TestVar<10^(-9)
   disp("The relation between deviators is verified")
end

disp(dash) %ex5
disp("Ex5: Stiffness tensor computation")

%ijkq=2323
I=eye(3);
for i=1:3
    for j=1:3
        for k=1:3
            for q=1:3
C(i,j,k,q)=E/(1+nu)*(0.5*( I(i,k)*I(j,q)+I(j,k)*I(i,q))+(nu/(1-2*nu)*I(i,j)*I(k,q)));
            end
        end
    end
end
C
disp(dash)
disp("C(2,3,2,3) is equal to")
C(2,3,2,3)
disp(dash)
disp("Hello World")

\end{verbatim}

\end{document}